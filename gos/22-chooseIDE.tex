\chapter{ВЫБОР ИНСТРУМЕНТАЛЬНОЙ СИСТЕМЫ РАЗРАБОТКИ ПО}
	
	В ходе подготовки к написанию выпускной квалификационной работы были рассмотрены несколько инструментальных сред разработки на C/C++. В качестве критериев для выбора IDE были выбраны следующие:
	
	\begin{itemize}
		\item IDE должна поддерживать разработку графического интерфейса пользователя;
		\item кроссплатформанная разработка;
		\item удобство программирования.
	\end{itemize}
	
	\section{Code::Blocks}
	
	Свободная кроссплатформенная среда разработки. 
	
	Для разработки графического интерфейса необходимо использовать плагин wxSmith, который, по сути, является  wxWidgets
	RAD инструментом, то есть позволяет создавать оконные формы и прочие графические объекты, используя библиотеку wxWidgets (библиотека wxWidgets устанавливается отдельно). 
	
	Плюсы Code::Blocks:
	
	 \begin{itemize}
		 \item автозавершение кода;
		 \item просмотрщик классов;
		 \item быстрая система сборки (не требуются make-файлы);
		 \item поддержка параллельных сборок.
	 \end{itemize}
	 
	\section{Microsoft Visual Studio Express}
	
	
	Данная версия Visual Studio представляет собой набор урезанных средств
	разработки для языков Visual Basic, C\# и C++, и обозначается Microsoft как инструментальная среда разработки начального
	уровня для тех лиц, кто не занимается профессионально программированием
	(школьников, студентов, любителей и т.д.).
	
	Графический интерфейс и возможность создать оконные приложения присутствует, но возможность воспользоваться наработками компании в области оптимизации и рефакторинга кода почти отсутствует.
	
	\section{Qt Creator}
	
	Qt Creator является средой разработки для кроссплатформенного фреймворка Qt. Вследствие этого, нужно указать на следующие его возможности:
	\begin{itemize}
		\item интеграция дизайнера форм Qt и справочной системы Qt
		\item расширяемость (посредством плагинов)
		\item поддержка отладчиков GDB (отладка графический интерфейса) и CDB
	\end{itemize}
	
	Общее сравнение IDE приведено в таблице \ref{tab:ide_effect}.
	
	\begin{longtable}{|p{3.5cm}|p{3.5cm}|p{4cm}|p{3.5cm}|}
		% Вертикальная черта означает, что между полями должна быть вертикальная черта - разделитель
		% Заголовок таблицы на первой странице:
		\caption{Сравнение IDE\label{tab:ide_effect}}\\
		\hline % Вставляем горизонтальную линию
	Наименование показателей & Code::Blocks & Microsoft Visual Studio Express & Qt Creator \\
		\hline
		\endfirsthead % Всё, что расположено выше считается заголовком таблицы и отображается на первой странице
		% Для второй и последующих страниц подменяем наименование таблицы в соответствии с требованиями:
		\caption*{Продолжение таблицы \ref{tab:ide_effect}}\\
		\hline
		\multicolumn{1}{|c|}{1} & \multicolumn{1}{|c|}{2 } & \multicolumn{1}{|c|}{3} & \multicolumn{1}{|c|}{4}\\
		\endhead % Всё что выше будет вставляться как заголовок на 2 и последующих страницах
		\hline
		
		Лицензия & \multicolumn{1}{c|}{GPL} & \multicolumn{1}{c|}{Freeware} & \multicolumn{1}{c|}{GPL} \\
		\hline
		Кроссплат\-форменность & \multicolumn{1}{c|}{Да} & \multicolumn{1}{c|}{Нет} & \multicolumn{1}{c|}{Да}  \\
		\hline
		Разработка GUI & \multicolumn{1}{c|}{Да\footnote{С использованием плагина wxSmith}} & \multicolumn{1}{c|}{Да} & \multicolumn{1}{c|}{Да} \\
		
		\hline
	\end{longtable}
	
	
	Исходя из приведенных выше сведений, было принято решение, что приложение будет написано в среде Qt Creator с использованием кроссплатформенного фреймворка Qt на языке C++. Это позволит сократить время на разработку, избежать написания большого количества кода, посредством использования готовых библиотек. Использование библиотеки Qt также поможет создать удобный интерфейс, отличный от стандартного.