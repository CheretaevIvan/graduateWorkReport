	\chapter{ТЕХНИЧЕСКОЕ ЗАДАНИЕ}
	
	\section{Общие сведения}
	
	\subsection{Полное наименование системы и ее условное обозначение}
	
	Программное обеспечение для автоматизации конфигурирования сетевого оборудования Cisco. Условное обозначение – АКСО\cite{gost-34602}.
	
	\subsection{Краткая характеристика области применения}
	
	АКСО предназначено для автоматизации конфигурирования сетевого оборудования Cisco путем предоставления пользователю графического интерфейса с последующим получением набора команд, необходимых для внедрения изменений внесенных пользователем.
	
	\subsection{Перечень документов, на основании которых создается система}
	
	\begin{enumerate}
		\item Приказ ректора УрФУ № \_\_\_\_\_\_ от "\_\_\_"\_\_\_\_\_\_\_\_\_\_ \_\_\_\_\_г.
	\end{enumerate}
		
	\subsection{Перечень документов, на основании которых устанавливается порядок оформления и предъявления результатов работ}
	
	\begin{enumerate}
		\item ГОСТ 7.32-2001. Система стандартов по информации, библиотечному и издательскому делу. Отчет о научно-исследовательской работе. Структура и правила оформления\cite{gost-732}.
%		\item ГОСТ 34.602-89 Информационная технология. Комплекс стандартов на автоматизированные системы. Техническое задание на создание автоматизированной системы.
%		\item оформление ВКР !!!!!
%		\item  Методические указания по выполнению выпускной квалификационной работы бакалавра техники и технологий по направлению "Информатика и вычислительная техника" / сост. А.Б.Николаев. – Москва: МАДИ, 2010. – 17 с.
		\item Соколов С.С. Рекомендации по оформлению курсовых, выпускных и дипломных проектов (работ). Электронные методические указания - Екатеринбург: Изд. УрФУ, 2010. - 38 с.
	\end{enumerate}
	
	
	\section{Назначение разработки}
	
	\subsection{Функциональное назначение}
	
	Функциональным назначением АКСО является предоставлению пользователю удобных инструментов для облегчения конфигурирования сетевого оборудования от компании Cisco Systems, таких как маршрутизаторы и/или коммутаторы.
	
	\subsection{Эксплуатационное назначение}
	
	АКСО предназначено для автоматизации процесса конфигурирования сетевого оборудования с целью получения исходного текста конфигурации либо автоматического применения внесенных с помощью АКСО изменений на сетевом оборудовании.
	
	\section{Требования к программному средству}
	
	\subsection{Требования к функциям, выполняемым системой }
	
	\begin{enumerate}
%		\item Разработать структуру классов, хранящую информацию о сетевых устройствах.
		
%		\item Наличие основных моделей маршрутизаторов и коммутаторов.
		
		\item Программа должна обеспечивать генерацию команд конфигурации, обеспечивающих:
		\begin{itemize}
			\item начальную настройку оборудования;
			\item настройку интерфейсов коммутатора/маршрутизатора;
			\item настройку VLAN'ов на коммутаторах;
			\item настройку DHCP на маршрутизаторах;
			\item настройку NAT на маршрутизаторах;
			\item настройку ACL(обычные и расширенные) на маршрутизаторах.
		\end{itemize}
		
		\item Наличие редактора, позволяющего добавлять и редактировать шаблоны.  Редактор должен обеспечивать возможность поиска в списке имеющихся моделей. 
%		Так же при помощи данного редактора должны производиться операции визуального изменения конфигурации для выбранного модели сетевого оборудования.

		\item Возможность выбора шаблона и ввода необходимых параметров в шаблон.		
		
%		\item Наличие возможности экпорта/импорта выбранной конфигурации в виде XML-файла, содержащего все необходимые сведения.
		
		\item Наличие возможности сохранения сгенерированной конфигурации в виде текстового файла, содержащего команды конфигурирования
		
		\item Наличие справочной подсистемы. Справка должна содержать краткую информацию о системе и ее возможностях, описание действий пользователя и получаемых результатов при работе с программным обеспечением.
	\end{enumerate}
	
	\subsection{Требования к надежности функционирования и безопасности}
	
	Надёжность системы должна обеспечивать работоспособность в течение всего срока эксплуатации при бесперебойном питании ЭВМ. Программное обеспечение не должно содержать явных логических ошибок и функционировать без сбоев.	
	
	\subsection{Требования к информационной и программной совместимости}
	
	АКСО должна иметь возможность функционировать под управлением различных операционных систем (Windows, Linux и т.д.).
	
	\subsection{Требования к аппаратному обеспечению}
	
	\begin{itemize}
		\item процессор Intel Pentium 2-4;
		\item оперативная память RAM не менее - 256 мб;
		\item свободное место на диске - не менее 80 мб;
		\item монитор;
		\item клавиатура;
		\item манипулятор мышь;
		
	\end{itemize}
	
	\subsection{Требования к исходным кодам и языкам программирования}
	
	Исходные коды программного средства должны быть реализованы на языке C++. В качестве интегрированной среды разработки программы должна быть использована среда Qt Creator.
	
	\subsection{Специальные требования}

	Программа должна обеспечивать взаимодействие с пользователем (оператором) посредством графического пользовательского интерфейса.
	
	\section{Стадии и этапы разработки}
	
	\subsection{Стадии разработки}
	
	Разработка должна быть произведена в три стадии\cite{iso-12207}:
	\begin{enumerate}
		\item Разработка технического задания;
		\item Рабочее проектирование;
		\item Внедрение;
	\end{enumerate}
	
	\subsection{Этапы разработки}
	На стадии рабочего проектирования должны быть выполнены перечисленные ниже этапы работ:
	\begin{enumerate}
		\item разработка АКСО; 
		\item разработка программной документации; 
		\item испытания АКСО.
	\end{enumerate}
	
	На стадии внедрения должен быть выполнен этап разработки - подготовка АКСО.
	
	\subsection{Содержание работ по этапам}
	
	На этапе разработки АКСО должна быть выполнена работа по программированию (кодированию) и отладке программного обеспечения (АКСО).
	
	На этапе разработки программной документации должна быть выполнена разработка программных документов в соответствии с требованием п. \ref{subsection:documentation} настоящего технического задания.
	
	На этапе испытаний АКСО должны быть выполнены перечисленные ниже виды работ:
	
	\begin{enumerate}
		\item проверка выполнения заданных функций АКСО;
		\item выявления и устранения недостатков в АКСО и программной документации; 
		\item корректировка АКСО и программной документации по результатам тестирований.
	\end{enumerate}
	
	На этапе подготовки АКСО должна быть выполнена работа по подготовке программного средства и программной документации для эксплуатации.
	
	\section{Порядок защиты и контроля}
	
	Защита осуществляется перед Государственной аттестационной комиссией (ГАК), утвержденной приказом ректора.
	
	\section{Требования к программной докуменации}
	\subsection{Предварительный состав программной документации}
	\label{subsection:documentation}
	Предварительный состав программной документации должен включать в себя\cite{gostr-9294}:
\begin{enumerate}
	\item техническое задание;
%	\item текст программы;
	\item описание программы;
	\item пояснительную записку\cite{methodVKR,methodVKRUrFU};
	\item руководство пользователя.
\end{enumerate}
	
	
	

	
	\section{Источники разработки}
	\begin{enumerate}
%		\item ГОСТ 19.201-78. Техническое задание, требования к содержанию и оформлению. 
%		\item ГОСТ 19.102-77 ЕСПД. Стадии разработки. 
%		\item ГОСТ 19.104-78 ЕСПД. Основные надписи. 
%		\item ГОСТ 19.105-78 ЕСПД. Общие требования к программным документам. 
%		\item ГОСТ 19.106-78 ЕСПД. Требования к программным документам, выполненным печатным способом. 
%		\item ГОСТ 28195-89. Оценка качества программных средств. Общие положения.
%		\item ГОСТ 19.781-90. Обеспечение систем обработки информации программное. Термины и определения
%		\item  Методические указания по выполнению выпускной квалификационной работы бакалавра техники и технологий по направлению "Информатика и вычислительная техника" / сост. А.Б.Николаев. – Москва: МАДИ, 2010. – 17 с.
%		\item Методические указания к выполнению выпускных бакалаврских работ по направлению 552800 «Информатика и вычислительная техника» по специальности 230101 65 «Вычислительные машины, комплексы, системы, сети»/Под общей ред.В.А.Бархоткина. - Москва:МИЭТ, 2006. - 15с.
%		\item Соколов С.С. Рекомендации по оформлению курсовых, выпускных и дипломных проектов (работ). Электронные методические указания - Екатеринбург: Изд. УрФУ, 2010. - 38 с.


%		\item ГОСТ 19.102-77 ЕСПД. Стадии разработки.
%		\item ГОСТ 19.105-78 ЕСПД. Общие требования к программным доку-
%		ментам.
%		\item ГОСТ 19.106-78 ЕСПД. Требования к программным документам,
%		выполненным печатным способом.
		\item ГОСТ 34.602-89 Информационная технология. Комплекс стандартов на автоматизированные системы. Техническое задание на создание автоматизированной системы.
		\item ISO/IEC 9294-93 (ГОСТ Р) Информационная технология. Руководство по управлению документированием программного обеспечения.
		\item ISO/IEC 12207:2008 (ГОСТ Р) Системная и программная инженерия. Процессы жизненного цикла программных средств.
		\item ISO/IEC 9126:1991 (ГОСТ Р) Информационные технологии. Оценка программного продукта. Характеристики качества и порядок их применения.
	\end{enumerate}