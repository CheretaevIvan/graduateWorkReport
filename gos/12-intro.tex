\Introduction

Во второй половине прошлого века в связи с «рождением» первых
вычислительных сетей произошла очередная научно-техническая революция. Появилась возможность начать использование рассредоточенной
обработки данных, обширно использовать для автоматизации различных
видов деятельности новые технологии. 



В наше время происходит активное применение сетевых решений во многих сферах деятельности. В условиях производства, на различных предприятиях,
в офисах компаний, различных фирмах и учреждениях отдельно стоящий, не подключенный к сети компьютер является большой редкостью. Если же подобной сети в учреждении нет или она плохо развита, от системных администраторов данной организации либо от специалистов компании, предоставляющей услуги в области сетевых технологий, требуется спроектировать и настроить сеть, удовлетворяющую потребностям клиента.

Важным моментом при этом является настройка сетевого оборудования. Одним из представителей компаний-производителей сетевого оборудования является Cisco Systems. Cisco Systems — компания-производитель
сетевого оборудования, основанная в 1984. Сначала компания производила маршрутизаторы, но затем значительно расширила ассортимент своей продукции. В настоящее время она производит коммутаторы, маршрутизаторы,
IP-телефоны, программное обеспечение для своего оборудования.



Специалистам, настраивающим оборудование Cisco, часто приходится использовать однообразные наборы команд конфигурации, порой отличающимися небольшим набором параметров, такими как IP-адрес устройства, идентификатор VLAN или диапазон IP-адресов при настройка DHCP.

% --- комплекс действий, для которых необходимо качество и точность выполнения. И, одновременно с этим, за прошедшее время настройка оборудования превратилась в 

Соответственно, необходимо решение, которое поможет быстро и легко настроить оборудование Cisco путем создания конфигурационных файлов для данного оборудования на основании различных шаблонов и введенных пользователем данных.