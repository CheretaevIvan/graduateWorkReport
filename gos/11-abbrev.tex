\chapter*{ОПРЕДЕЛЕНИЯ, ОБОЗНАЧЕНИЯ И СОКРАЩЕНИЯ}
\addcontentsline{toc}{chapter}{ОПРЕДЕЛЕНИЯ, ОБОЗНАЧЕНИЯ И СОКРАЩЕНИЯ}

\makeatletter
\setlength{\@fptop}{0pt}
\makeatother

\begin{table*}[ht!]
	\begin{tabular}{p{0.22\textwidth}p{0.78\textwidth}}
%		\label{tab:tabular}
		Маршрутизатор & Специализированный сетевой компьютер, имеющий два или более сетевых интерфейсов и пересылающий пакеты данных между различными сегментами сети \\
		
		Коммутатор & Устройство, предназначенное для соединения нескольких узлов компьютерной сети в пределах одного сегмента сети\\

		ПО & Программное обеспечение\\
		
		Cisco IOS & Internetwork Operating System — Межсетевая Операционная Система --- программное обеспечение, используемое в маршрутизаторах и сетевых коммутаторах Cisco\\
		
		VLAN & Virtual Local Area Network --- группа устройств, имеющих возможность взаимодействовать между собой напрямую на канальном уровне, хотя физически при этом они могут быть подключены к разным сетевым коммутаторам\\
		
		IDE & Integrated development environment --- комплекс программных средств, используемый программистами для разработки программного обеспечения\\
		
		TCP/IP & Transmission Control Protocol / Internet Protocol --- набор сетевых протоколов передачи данных, используемых в сетях, включая сеть Интернет\\
		
		IP &  Internet Protocol --- маршрутизируемый протокол сетевого уровня стека TCP/IP\\
		
		DHCP & Dynamic Host Configuration Protocol --- это сетевой протокол, позволяющий компьютерам автоматически получать IP-адрес\\
		
		
		
	\end{tabular}
	
\end{table*}

\makeatletter
\setlength{\@fptop}{0pt}
\makeatother

\begin{table*}[ht!]
	\begin{tabular}{p{0.22\textwidth}p{0.78\textwidth}}
		
		OSPF & Open Shortest Path First --- протокол динамической маршрутизации, основанный на технологии отслеживания состояния канала и использующий для нахождения кратчайшего пути алгоритм Дейкстры.\\
		
		
		EIGRP & Enhanced Interior Gateway Routing Protocol --- проприетарный протокол динамической маршрутизации класса "вектор расстояния" (Distance Vector), в котором информация передается от соседа к соседу, каждый следующий выбирает только лучший маршрут, отдаваемый соседу.\\	
		
		NAT &  Network Address Translation --- механизм в сетях TCP/IP, позволяющий преобразовывать IP-адреса транзитных пакетов\\
		
		ACL & Access Control List --- список правил, определяющих порты служб или имена доменов, доступных на узле или другом устройстве\\
		
		
		
			
	\end{tabular}	
\end{table*}
		
%%% Local Variables:
%%% mode: latex
%%% TeX-master: "rpz"
%%% End:
